\section{Introduction} \label{introduction} 

%length: 1 page including headings
%resposible: Lu & Gui

%content:
%introduction and motivation

No matter what programming language students are learning they need a lot of practice in order to get good at it. Also, the practice is more effective if it is guided by useful formative feedback. However, teacher feedback is costly and slow. As a result, there are many coding tools that offer programming exercises and provide automated feedback
on student solutions. According to a recent review~\cite{Keuning:2016}, most of these tools support imperative programming languages and less than 10\% of them support functional programming. 

Many coding tools not only generate automated formative feedback, but they also log large amounts of educational data. As argued by previous work~\cite{Ihantola:2015}, these logs could feed educational data mining in order to better understand student behaviour. However, the majority of the studies are conducted within a single institution and a single course lowering the reliability of the results.

The goal of this paper is to investigate how students with different backgrounds use an online coding tool and how this influences their behaviour when learning functional programming. We designed an observational study to compare how CS1 students (with no previous background in programming), and CS3 students (with two previous semesters of imperative programming courses) use an online coding tool to learn Haskell. We piloted CS lessons in two different universities. Our courses focused on modularity and function composition, conditionals, pattern matching, lists, tuples, and recursion. %We analysed student behaviour while using a web-based coding tool and compared male and female performance. 
The main contributions of this paper are:
\begin{itemize}
\setlength\itemsep{0pt}
\setlength{\parskip}{0pt}
\item Analyse how students with different background learn a functional programming language. 
\item Introduce a web-based coding tool that generates automated formative feedback for Haskell 
\item Evaluate gender and background differences in indicators of learning and engagement in CS1 and CS3. 
\end{itemize}

We begin the paper by describing the online coding tool for Haskell that we use, called Mumuki, and its rationale. Then, we address the study design followed by our findings. We analyse student performance discriminated by their background and gender. Moreover, we discuss common errors in students learning functional programming in our courses. Finally, we summarise previous work on educative coding tools to teach functional programming. We close this paper with conclusions and implications for teaching Haskell at different university levels.



%list of contributions of the paper
%plan of the paper





