\section{Conclusions} \label{conclusion}

%length: 1 page including references and acks
%resposible: everybody

%content:

Through two university interventions focusing on learning a functional programming language with the assistance of an online coding tool we found that students benefit from the availability of such tool and that they assess it positively. In spite of these results and previous work, the adoption of these tools is not widespread and they are usually used only in the university where they were created (as discussed in~\cite{brusilovsky2014increasing}). We present in this paper two interventions in two different universities where the teachers adopt the tool in spite of not being the developers.
CS1 students seem to benefit more from using Mumuki than CS3 students. Their dropout rate was improved more than in CS3 and their assess the usefulness of the tool more positively although it was harder for them to complete the exercises. Differently from previous work we found no gender differences neither in performance nor in attitude. 
 
There are several implication of our findings. 1) Based on our research and on previous work we suggest that student dropout rate may diminish when using educative coding tools when learning to program. In our study this is particularly so in CS1. 2) Students can transfer programming skills from imperative languages into functional languages but they also can learn functional languages as their first programming language without difficulties. 3) The logs of these online tools can serve to explore the most common errors and explain them in-class. Thorough in-class explanations diminishes the need of better automated feedback but are time consuming. We will explore the automatic generation of better feedback on errors with data-driven methods in future work. 

%After finishing this course our students continued using Mumuki to learn other programming languages on their own and 95\% said that they wanted to continue using it in the future.  

 
